\documentclass[11pt]{article}
\usepackage{times}
\usepackage{fullpage}
\usepackage{hyperref}
\usepackage{fancyhdr}

\title{%
    {\Large Portion Mate \par}
    {\small Daily Food Portion Tracker \par}
    \vspace{0.5em}
    {\LARGE Status Report}
}
\author{Inesh Bose (2504266B)}
\date{}

\begin{document}

\maketitle

\newcommand{\tightlist}[0]{\setlength\itemsep{0.2em}}


\section{Proposal}\label{proposal}

\subsection{Motivation}\label{motivation}

Portion tracking apps exist already, however, most of them are not quite dedicated for diet and portions that follow a guide. This application is being developed mostly since it is part of a required course - however, there is a lot of keen interest in learning and developing this individually using best practices and make a quality product with a good user interface.

\subsection{Aims}\label{aims}

This project will develop a mobile and web application for users that allows them to keep a track of their daily portion count for each food item/group, that can also be based on a diet plan such as The Eatwell Guide. The user experience should be very simple and easy by having easy input (such as ticking off numbers) and reminders, since portion tracking applications require commitment and dedication from the user, to keep their data timely and accurate, which might be inconvenient at times.

\section{Progress}\label{progress}

\begin{itemize}
    \tightlist
    \item \textbf{Language and GUI framework chosen}: project will be implemented in Python using \href{https://www.djangoproject.com/}{Django}, and \href{https://www.typescriptlang.org/}{TypeScript}/JavaScript using \href{https://reactnative.dev/}{React Native} for GUI development
    \item \textbf{Research conducted}: food guides, strategies, \href{https://github.com/ineshbose/portion-mate/wiki/Requirements}{user \& application requirements}, similar applications, and technologies / frameworks were studied prior to starting implementation
    \item \textbf{Repository and project setup}: a very particular and strict \href{https://github.com/ineshbose/portion-mate/tree/develop/.github}{automated and integrated system} has been developed for version control, linting, tests, dependencies, deployment, and issue tracking
    \item \textbf{Applications started and wireframes drafted}: the base applications for the \href{https://github.com/ineshbose/portion-mate/tree/develop/src/backend}{backend} (\href{https://github.com/ineshbose/portion-mate/tree/develop/src/backend/main}{\texttt{main}} and \href{https://github.com/ineshbose/portion-mate/tree/develop/src/backend/rest}{\texttt{rest}}) and \href{https://github.com/ineshbose/portion-mate/tree/develop/src/frontend}{frontend} have been created, along with a \href{https://github.com/ineshbose/portion-mate/wiki/Wireframes}{UI wireframe}
    \item \textbf{Software architecture and entities outlined}: the \href{https://github.com/ineshbose/portion-mate/blob/develop/src/backend/main/models.py}{database models} have been designed, and the architecture uses \href{https://reactnative.dev/docs/components-and-apis}{Component-based frontend} framework and a \href{https://www.django-rest-framework.org/}{REST API backend}
\end{itemize}

\section{Problems and risks}\label{problems-and-risks}

\subsection{Problems}\label{problems}

\begin{itemize}
    \tightlist
    \item While there is experience with \href{https://reactjs.org/}{React} and the frontend framework being based off of React, it is, however, still quite different and does not have a component library (such as \href{https://react-bootstrap.github.io/}{React-Bootstrap}).
    \item React Native app \href{https://reactnative.dev/docs/performance}{may perform slow} and be heavy on client side compared to other frameworks.
\end{itemize}

\subsection{Risks}\label{risks}

\begin{itemize}
    \tightlist
    \item User might have different, specific dietary restrictions and/or allergies.\newline \textbf{Mitigation}: will offer customisation ability to users.
    \item Unclear how to evaluate success of the project.\newline\textbf{Mitigation}: will do research to investigate success of diet plans in portion tracking.
    \item Deployment of application and sharing on application stores/markets for platforms.\newline\textbf{No clear mitigation available at this stage.}
\end{itemize}

\section{Plan}\label{plan}

\subsection{Semester 2}

\begin{itemize}
    \tightlist
    \item
      \textbf{Week 1-2}: develop user interface\newline
      \textbf{Deliverable}: complete designing frontend React application
    \item
      \textbf{Week 3}: integrate database and API\newline
      \textbf{Deliverable}: application interacting with the Django backend
    \item
      \textbf{Week 4-5}: implement, test, debug\newline
      \textbf{Deliverable}: a functional application with features tested and added as new ideas
    \item
      \textbf{Week 6}: research on how to best evaluate performance of final system\newline
      \textbf{Deliverable}: detailed evaluation plan, with participant numbers, information sheet and analysis plan
    \item
      \textbf{Week 7-9}: final implementation and improvements to application \& performance\newline
      \textbf{Deliverable}: polished software ready, passing basic tests, ready for evaluation stage
    \item
      \textbf{Week 9}: evaluation experiments run\newline
      \textbf{Deliverable}: quantitative (and qualitative) measures of usability \& effectiveness for at least ten users
    \item
      \textbf{Week 8-10}: Write up\newline
      \textbf{Deliverable}: first draft submitted to supervisor two weeks before final deadline
    \end{itemize}

\section{Ethics}

This project will require evaluation with human users. These will be studies where users would install the application on their personal devices which will also require their personal/athletic data such as height and weight, but in no way will be used for any analysis or recognition of individuals -- I have verified that the experiments I plan to do comply with the \href{https://www.dcs.gla.ac.uk/ethics}{Ethics Checklist}.

\end{document}
