%==============================================================================
% PortionMate: app for tracking daily food portion counts (ID: 14081)
%  Owner/Supervisor: Oana Andrei
%  * Suitable as a Software Engineering project.
%  * Project may require participation from people other than the student and the supervisor as part of the evaluation.
%
% Description
% Design and develop a mobile app to help the user keep track of their daily
% portion count from each food group based on the Eat(h)well Guide [1,2].
% The app should be easy to use to tick off number of portions on a daily chart.
% The app would set daily goals for the portion counts for people to achieve and
% allow record of additional portions. General stats per week/month/year should
% also be included. Further discussion about possible features is required at the
% start of the project
%
%  [1] https://www.gov.uk/government/publications/the-eatwell-guide
%  [2] https://www.nhs.uk/live-well/eat-well/the-eatwell-guide/
% Special Requirements :
% No ethics checklist or approval required.
%
%==============================================================================
%


\documentclass{l4proj}

% The bib files are imported from Zotero API
% https://api.zotero.org/users/:userId/collections/:collectionId/items?limit=100&format=bibtex
% However, the pagination only allows 100 items at-most, so it has also been split into separate files
\addbibresource{l4proj.bib}
\addbibresource{tech.bib}   % filtered tag=package
\addbibresource{repo.bib}   % filtered tag=repository
\nocite{*}

\graphicspath{{images/}}

\begin{document}

%==============================================================================
% METADATA

\title{%
    \includesvg[width=200pt]{title_logo.svg}\par
    {\Huge Portion Mate \par}
    {\Large Daily Food Intake Tracker \par}
    \large
}
\author{Inesh Bose}
\date{Academic Session 2021-22}

\maketitle

%==============================================================================
% ABSTRACT
% Every abstract follows a similar pattern. Motivate; set aims; describe work; explain results.

%    ``XYZ is bad. This project investigated ABC to determine if it was better.
%    ABC used XXX and YYY to implement ZZZ. This is particularly interesting as XXX and YYY have
%    never been used together. It was found that
%    ABC was 20\% better than XYZ, though it caused rabies in half of subjects.''

\begin{abstract}
    The Eatwell Guide is a recommendation given by Public Health England on having a balanced diet. Following and maintaining a diet may be difficult for people, and logging can help them keep track of their plan. However, the task of logging itself can also require a lot of commitment which can cause people inconvenience and make them give it up.

    This project aims to develop an application that solves these problems of maintaining a diet plan; this was achieved through months of research and development into users, food and logging, in order to create a fully-integrated and convenient system that does not add a burden onto users, but rather motivates them to adhere to their plan. Considering the time-frame, this project also leaves development open-ended with scope for future work that can be passed over to or taken over by developers who can make use of the open-source side projects created and best practices followed by the system without great difficulty.

    To confirm the usability of the application, evaluation was carried out through a survey where participants could share their thoughts with the help of System Usability Scale that also in-turn helps in analysing the results to a score that deemed the system as acceptable.
\end{abstract}

%==============================================================================
% ACKNOWLEDGEMENTS

\begin{acknowledgements}
    My sincere gratitude goes to my supervisor, Dr. Oana Andrei, for being the most understanding and providing guidance to this fourth year student who aimed high and stressed endlessly for this project. Most of the development has also been made possible because of my previous/current employers who have given me incredible amount of precious experience and knowledge in the area of my interest \& passion - computing and software development.

    The participants who helped evaluate the project and provide important feedback. The communities and developers on platforms like GitHub, StackOverflow, Reddit, Discord, and many-many open-source projects that have given an immeasurable amount of help in providing solutions to problems that would take months. My friends and, most importantly, my family who make me who I am today with all their support and providing me with the opportunity to come to Glasgow to learn from one of the greatest educational institutions in the world.

    I would also like to thank \textbf{you}, the reader, for taking the valuable time to take a look at this dissertation, which already means the world to me. Thank you.
\end{acknowledgements}

%==============================================================================
% EDUCATIONAL REUSE CONSENT FORM
% If you consent to your project being shown to future students for educational purposes
% then insert your name and the date below to  sign the education use form that appears in the front of the document.
% You must explicitly give consent if you wish to do so.
% If you sign, your project may be included in the Hall of Fame if it scores particularly highly.
%
% Please note that you are under no obligation to sign
% this declaration, but doing so would help future students.
%

\def\consentname {Inesh Bose} % your full name
\def\consentdate {1 April 2022} % the date you agree
%
\educationalconsent

%==============================================================================

\tableofcontents

\newpage
\listoffigures
\listoftables

%==============================================================================
%% Notes on formatting
%==============================================================================
% The first page, abstract and table of contents are numbered using Roman numerals and are not
% included in the page count.
%
% From now on pages are numbered
% using Arabic numerals. Therefore, immediately after the first call to \chapter we need the call
% \pagenumbering{arabic} and this should be called once only in the document.
%
% Do not alter the bibliography style.
%
% The first Chapter should then be on page 1. You are allowed 40 pages for a 40 credit project and 30 pages for a
% 20 credit report. This includes everything numbered in Arabic numerals (excluding front matter) up
% to but excluding the appendices and bibliography.
%
% You must not alter text size (it is currently 10pt) or alter margins or spacing.
%
%
%=============================================================================
%
% IMPORTANT
% The chapter headings here are **suggestions**. You don't have to follow this model if
% it doesn't fit your project. Every project should have an introduction and conclusion,
% however.
%
%=============================================================================

\chapter{Introduction}
\pagenumbering{arabic} % reset page numbering. Don't remove this!
\subfile{chapters/01_introduction}

\chapter{Background}
\subfile{chapters/02_background}

\chapter{Requirements}
\subfile{chapters/03_requirements}

\chapter{Design}
\subfile{chapters/04_design}

\chapter{Implementation}
\subfile{chapters/05_implementation}

\chapter{Evaluation}
\subfile{chapters/06_evaluation}

\chapter{Conclusion}
\subfile{chapters/07_conclusion}

%=============================================================================
%
%
%=============================================================================
% APPENDICES

% Typical inclusions in the appendices are:

% -   Copies of ethics approvals (required if obtained)

% -   Copies of questionnaires etc. used to gather data from subjects.

% -   Extensive tables or figures that are too bulky to fit in the main
%     body of the report, particularly ones that are repetitive and
%     summarised in the body.

% -   Outline of the source code (e.g. directory structure), or other
%     architecture documentation like class diagrams.

% -   User manuals, and any guides to starting/running the software.

% **Don’t include your source code in the appendices**. It will be
% submitted separately.

\begin{appendices}

\chapter[Figures]{Figures \& Graphics}
\subfile{chapters/appendix/a_figures}

\chapter{Code Listings}
\subfile{chapters/appendix/b_code_listings}

\chapter{Extra Discussion}
\subfile{chapters/appendix/c_extra_discussion}

\end{appendices}

%=============================================================================
% BIBLIOGRAPHY

% The bibliography always appears last, after the appendices.

\renewcommand{\bibsection}{\chapter*{\bibname}}

\printbibliography[notkeyword=repository]

\DeclareFieldFormat{title}{Portion Mate. GitHub. \code{#1}}
\printbibliography[keyword=repository,heading=subbibliography,title={Repository}]

\DeclareFieldFormat{title}{\textit{#1}}
\printbibliography[notcategory=cited,heading=subbibliography,title={Further Reading}]

\end{document}
